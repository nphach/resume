\documentclass[letterpaper]{article}
\usepackage[margin=1in]{geometry}
\usepackage{nopageno}
\usepackage{fontawesome5}
\usepackage{xcolor}
\usepackage{xhfill}
\usepackage{setspace}
\usepackage[T1]{fontenc}
\usepackage[sfdefault]{inter}
\usepackage{titlesec}
\usepackage{hyperref}

\definecolor{accent}{RGB}{63, 143, 180} % Boston blue
\newcommand{\infotab}{\quad\textbar\quad}
\newcommand{\dfill}{\xdotfill[0.5ex]{.4pt}}

\linespread{1.5}

\let\oldfaIcon\faIcon
\renewcommand{\faIcon}[1]{{\color{accent}\oldfaIcon{#1}}}

\let\oldsection\section
\renewcommand{\section}[1]{\vspace{5mm}\noindent{\large\bfseries\color{accent} #1}\enspace\dfill\par}

\titlespacing*{\section}
  {0pt}
  {10pt}
  {2pt}

\begin{document}
    \begin{center}
        % header: name
        {\huge \textbf{Nikki Phach}}
        \vspace{5mm}

        %header: contact info
        \mbox{} \dfill\enspace
            \faIcon{map-marker-alt} Boston, MA \infotab
            \faIcon{phone-alt} (401) 585-3212 \infotab
            \faIcon{envelope} nikkiphach@gmail.com
        \enspace \dfill
    \end{center}

    \vspace{5mm}

    % cover letter
    \noindent
    Dear hiring management,
    \vspace{5mm}

    Hello! My name is Nikki Phach, and I recently earned my BSc in CS from the University of Massachusetts Boston. I pursued Computer Science because I enjoy mathematics and problem-solving, and saw opportunity to be able to apply myself creatively. I'm now looking to jumpstart my career in software development.
    
    During my studies, I've gained proficiency in Python, C and Java (with some experience in Racket and Assembly) across a variety of projects. I also gained an interest in AI technology which led me to my capstone project 'Image Analysis', where I worked in a team for a real-world client. Our goal was to develop a skin detection algorithm intended for dermatology applications, avoiding bias and discrimination of the subject. My contributions included researching skin detection methods related to color analysis, programming, and helped to create a subset of expected output images to be used for testing. I had refactored an existing algorithm for ease of use and readability, and added a class that calculated performance metrics for the algorithm to be used for analysis; I also implemented asynchronous image processing to decrease program runtime. Our algorithm yielded a final accuracy of 97.9\% across a dataset of 1000 photos! (For context, other published algorithms averaged around 78\% accuracy.)

    Independently, I've also been learning JavaScript, HTML and CSS. I used these skills to build my personal portfolio website, a React application with a responsive design that features a short biography, my past projects, a working contact form and interactive animation effects. My progress on this project can be viewed at \href{https://nphach.github.io/}{https://nphach.github.io/}. I'm always enthusiastic about learning new technologies and still continue to study machine learning in my free time. Some of my other hobbies include crochet, cooking and Japanese language study.

    Thank you for your consideration. If my skills align with your organization's goals, I'm available to meet and discuss my background further. Looking forward to hearing back from you!
    

    \vspace{5mm}
    Best regards,

    Nikki Phach

\end{document}